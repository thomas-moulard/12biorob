\section{Introduction}

\IEEEPARstart{F}{ollowing} a planned trajectory on a robot while
compensating execution errors has been extensively studied in the 90's
for mobile robots \cite{91icra.samson,
  98deLucaOrioloSamson}. Surprisingly, this issue has not been
explicitely addressed in the literature concerning navigation for
legged robots, although these machines are also prone to execution
errors while moving.

\textcolor{red}{Previous experiments such as
  \cite{11humanoids.baudouin} illustrate how imprecise can be
  trajectory following on a humanoid robot. After executing a five
  meters long trajectory, the difference between the planned and real
  position can reach $0.4 \textrm{m}$. Such an error cannot be ignored
  anymore and invalidate the whole planning stage. Therefore, solving
  this issue is crucial and will allow the achievement of complex
  movements where a high precision is needed. For instance, obstacle
  crossing is only feasible at the beginning of the trajectory where
  the drift is not too important. This paper objective is to provide a
  generic framework for robust trajectory following on a humanoid
  robot.  }

One way of indirectly tackling the problem consists of regularly
replanning the motion of the robot from its current configuration to
the goal after localizing obstacles with respect to the robot. This
strategy enables the robot to be reactive to environment changes as
well as to execution errors~\cite{05humanoids.michel,
  06icra.MichelChestnut,10springer.chestnut}. On the other hand, it
requires short planning time and induces heavy CPU load. It might even
not be always be possible. Indeed most fast replanning schemes rely on
a simplified model~\cite{01icra.KajitaKanehiro} of the robot
neglecting momenta generated by the leg motions. These assumptions are
not met for small robots like Nao~\cite{wikipedia.nao} with a large
ratio of mass distributed in the legs and with a small CPU.

Moreover, to produce a really feasible movement, additional
constraints must be satisfied: no auto-collision should occur during
the movement for instance.


\begin{figure}[ht!]
  \begin{center}
    \includegraphics[width=.45\textwidth]{fig/exp.png}
  \end{center}
% 4cm in x and y
  \caption{HRP-2 robot following a trajectory in a constrained
    environment. In this experiment, the robot starting position is
    deliberately perturbed. During the execution, the correction algorithm
    automatically cancels the perturbation and the robot reaches the
    desired final position. \label{fig:following}}
\end{figure}



Due to all these factors, validating a complex movement remains a
computationally expensive operation. Therefore, an alternative
solution to online replanning and regeneration of the walking
trajectory such as \cite{11icra.dimitrov, 10ar.herdt,
  06icra.nishiwaki, 05humanoids.michel} is the continuous deformation
of walking trajectories.  This combination of dynamic trajectories and
high probability for the robot to enter in auto-collision makes naive
correction algorithm fail which is why it is important to define a
sound framework for trajectory following.


This paper presents a ``blink of an eye'' reshaping of the trajectory
associated with a generic method to follow trajectories on a humanoid
robot. These two features together provide a way to follow a
trajectory while compensating for errors during the movement
execution. This opens many possible applications such as moving in
extremely constrained environments in a reliable manner, going to
specific places of the environment precisely, etc. Most of the state
of the art demonstration of reactive pattern generators are, in fact,
open loop trajectories with no sensors feedback. This work has been
fully integrated into the LAAS/JRL planning and control frameworks and
a motion capture system has been used to close the loop and evaluate
the execution errors.

This allowed HRP-2 humanoid robot to perform precise and/or long
locomotion tasks where usual open loops approaches would have drifted
so much that the task would have failed.

\FloatBarrier

%%% Local Variables:
%%% ispell-local-dictionary: "american"
%%% LocalWords:
%%% End:
