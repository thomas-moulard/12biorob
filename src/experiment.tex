\section{Experimental results}
\label{exp}


The proposed method has been validated on HRP-2 using the following
scenarii: the robot walks in a constrained area cluttered with 3D
obstacles. The planned motion generates steps where HRP-2 walks over
obstacles to reach its goal.


This experiment uses the pattern generator described by
\cite{10icra.perrin} and the stack of tasks formalism
\cite{09icar.mansard} to implement the control scheme. The pattern
generator provides reference trajectories for the center of mass and
the feet. Trajectory following tasks for these particular bodies are
then inserted into the control framework. The solver will then realize
implicitly the inverse geometry computations required to recompute the
whole-body trajectory and simplifies the actual control scheme.


\begin{figure}[ht!]
  \begin{center}
    \begin{tikzpicture}
      \def\w{1.}
      \def\h{1.}

      \foreach \x in {0.,...,6.}
      {
        \draw (
        \x * \w, 0.)
        rectangle (
        \x * \w + \w, 0. + \h);
      }

      \foreach \x in {0.,...,2.}
      {
        \draw (
        \x * \w, \h)
        rectangle (
        \x * \w + \w, 2. * \h);
      }
      \foreach \x in {5.,...,6.}
      {
        \draw (
        \x * \w, \h)
        rectangle (
        \x * \w + \w, 2. * \h);
      }

      \foreach \x in {2.,...,4.}
      {
        \draw (
        \x * \w, 0.)
        rectangle (
        \x * \w + \w, -\h);
      }

      \filldraw[rounded corners, pattern=north east lines] (
      0.25, 0.4)
      rectangle (
      0.75,1.7);

      \filldraw[rounded corners, pattern=north west lines,rotate=-10] (
      0.25, 0.4)
      rectangle (
      0.75,1.7);

      \filldraw[rounded corners, pattern=north east lines] (
      0.25+6., 0.4)
      rectangle (
      0.75+6.,1.7);
      \filldraw[rounded corners, pattern=north west lines] (
      0.25+6., 0.4)
      rectangle (
      0.75+6.,1.7);

    \end{tikzpicture}
  \end{center}
  \caption{HRP-2 experiment scenario. On the left, the vertical
    rectangle is the planned starting point and the rotated rectangle
    is the real starting point. On the right, the goal point which is
    reached independently of the initial error. \label{fig:scenario}}
\end{figure}


Fig.~\ref{fig:following} illustrates the experiment on the real
robot. This experiment video is available on the
web\footnote{\mbox{\url{http://homepages.laas.fr/tmoulard/video/11humanoids-tmoulard.mp4}}}. Fig.~\ref{fig:scenario}
provides an overview of the scenario: the rectangle on the left
symbolizes the initial position and the rectangle on the right the
final position. The robot is allowed to walk on the tiles, i.e.\ the
boxes but not outside. The starting point is perturbed to trigger a
correction.


\FloatBarrier

%%% Local Variables:
%%% ispell-local-dictionary: "american"
%%% LocalWords:
%%% End:
